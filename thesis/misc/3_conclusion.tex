%%
% The BIThesis Template for Bachelor Graduation Thesis
%
% 北京理工大学毕业设计(论文)结论 —— 使用 XeLaTeX 编译
%
% Copyright 2020 Spencer Woo
%
% This work may be distributed and/or modified under the
% conditions of the LaTeX Project Public License, either version 1.3
% of this license or (at your option) any later version.
% The latest version of this license is in
%   http://www.latex-project.org/lppl.txt
% and version 1.3 or later is part of all distributions of LaTeX
% version 2005/12/01 or later.
%
% This work has the LPPL maintenance status `maintained'.
%
% The Current Maintainer of this work is Spencer Woo.
%
% Compile with: xelatex -> biber -> xelatex -> xelatex

\addcontentsline{toc}{chapter}{结~~~~论}
\chapter*{\vskip 10bp\textmd{结~~~~论} \vskip -6bp}

本文主要研究三维空间上纯角度单站无源跟踪定位算法,在满足可观测的条件下,总结了四种求解匀速运动目标轨迹的定位算法,并通过相关数值实验对算法进行了定位性能的分析.通过初步实验,根据100次Monte Carlo实验的结果,可以看出,AUE算法和MLE算法的性能较好,受噪声的影响比OLS算法和TLS算法小.TLS算法和OLS算法易受噪声影响,求解得到的结果的相对误差较大.通过对测量数据累积求解目标末时状态矢量分析,AUE算法和TLS算法可以利用较少的测量数据求解得到较为准确的结果,而且利用$300 \sim 400$组数据便可求解相对精确的结果.但是在计算的过程中,MLE算法的结果取决于初始值的选取,由于其它算法对目标速度的估计值准确对较差,使得MLE算法以此为初始值求解出现不收敛的情况,同时MLE算法的迭代产生的计算量较大,不易做为最优的求解算法.

通过对影响算法求解结果的因素分析可知,目标距离观测站越近,算法的求解结果就会越精确.初步判断为目标距离观测站较近时,观测所得的方位角及仰角数据就会变的稀疏,极差变大,即相邻的两组测量数据的差值变大,使得噪声的影响降低,从而使得求解结果更加精确.通过方位角及仰角对算法结果的影响实验可知,方位角及仰角间差值确实对算法结果产生较大的影响.同等条件下,AUE算法比其它受到影响的程度较小.当测量误差增大时,噪声在测量角度中占的比例变大,对测量结果产生较大影响,AUE算法的精确度和稳定性降低.

综上分析,对于纯角度单站无源跟踪定位,AUE算法表现的性能比其它三种算法好,受噪声影响比其它三种算法小.目标与观测站的距离越近,算法的精确度越高.方位角及仰角做为影响算法结果的重要因素,当测角数据的极差越大,即相邻测量数据的差值越大,噪声在测量值中所占的比例就会越小,算法求解结果就会更精确.不过由于噪声对目标运动方向的影响比对目标运动位置的影响大,所以会使得算法求解结果的速度相对误差比距离相对误差偏大.

