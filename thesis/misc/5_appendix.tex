%%
% The BIThesis Template for Bachelor Graduation Thesis
%
% 北京理工大学毕业设计(论文)附录 —— 使用 XeLaTeX 编译
%
% Copyright 2020 Spencer Woo
%
% This work may be distributed and/or modified under the
% conditions of the LaTeX Project Public License, either version 1.3
% of this license or (at your option) any later version.
% The latest version of this license is in
%   http://www.latex-project.org/lppl.txt
% and version 1.3 or later is part of all distributions of LaTeX
% version 2005/12/01 or later.
%
% This work has the LPPL maintenance status `maintained'.
%
% The Current Maintainer of this work is Spencer Woo.
%
% Compile with: xelatex -> biber -> xelatex -> xelatex

\addcontentsline{toc}{chapter}{附~~~~录}
\chapter*{\vskip 10bp \textmd{附~~~~录} \vskip -6bp}

\section*{A}
由式(2-4)得到
\begin{subequations}
	\begin{align}
		0=& r_x\sin\beta - r_y\cos\beta \\
		0=& r_x\sin\varepsilon\cos\beta+r_y\sin\varepsilon\sin\beta -r_z \cos\varepsilon
	\end{align}
\end{subequations}
令式(4-1a)等式两边同时乘上 $\sin\beta$,式(4-1b)等式两边同时乘上 $\cos\beta/\sin\varepsilon$,再将结果相加得到
\begin{equation}
	0=r_x - r_z \cos\beta\cot\varepsilon
\end{equation}
再令式(4-1a)等式两边同时乘上 $-\cos\beta$,式(4-1b)等式两边同时乘上 $\sin\beta/\sin\varepsilon$,将结果相加得到
\begin{equation}
	0=r_y - r_z\sin\beta\cot\varepsilon
\end{equation}
从而得到
\begin{equation}
	H \bm{X} = \bm{0}
\end{equation}

\setcounter{section}{1}
\section*{B}
考虑如下线性系统
\begin{equation}
	\begin{split}
		\dot{\bm{x}} =& A\bm{x} - \bm{w} \\
		\bm{z} =& H \bm{x}
	\end{split}	
\end{equation}
其中 $A,H$ 是给出的连续可微矩阵,$\bm{w}$ 是已知向量,$\bm{x}$ 是 $n$ 维向量,$\bm{z}$ 是测量向量,并且该系统中不存在噪声干扰,视为理想情况.
令 $C_0 = H, \bm{z}_0 = \bm{z}$,第一次微分可得
\begin{equation}
	\dot{\bm{z}}_0 = \dot{C}_0\bm{x} + C_0\dot{\bm{x}}
\end{equation}
将式(4-5)中 $\dot{\bm{x}}$ 代入上式,并令 $C_1=\dot{C}_0 + C_0A,\bm{z}_1=\dot{\bm{z}}_0 + C_0\bm{w}$ 可得
\begin{equation}
	\bm{z}_1 = C_1 \bm{x}
\end{equation}
对式(4-7)微分并重复上述过程可得
\begin{equation}
	\bm{z}_2 = C_2 \bm{x}
\end{equation}
其中 $C_2 = \dot{C}_1 + C_1 A,\bm{z}_2 = \dot{\bm{z}}_1 + C_1\bm{w}$,继续并重复上诉过程 $n-2$ 次得到一组独立的关于 $x$ 的线性方程
\begin{equation}
	\bm{Z} = C\bm{x}
\end{equation}
其中
\begin{equation}
	\begin{split}
		\bm{Z} =& [\bm{z}_0,\bm{z}_1,\cdots,\bm{z}_{n-1}]^T \\
		\bm{z}_0 =& \bm{z} \\
		\bm{z}_{i+1} =& \dot{\bm{z}}_{i} + C_i\bm{w} \quad i=0,1,\cdots\\
		C =& [C_0,C_1,\cdots,C_{n-1}]^T \\
		C_0 =& H \\
		C_{i+1} =& \dot{C}_i + C_i A \quad i=0,1,\cdots
	\end{split}
\end{equation}
对式(4-9)左乘 $C^T$ 可得
\begin{equation}
	C^T\bm{Z} = C^TC \bm{x}
\end{equation}
则当且仅当 $C^TC$ 是非奇异的,即 $\det[C^TC] \not \equiv 0$ 时,$\bm{x}$ 可以被唯一确定.若 $C^TC$ 是奇异的,则
\begin{equation}
	\bm{x} = (C^TC)^*C^T Z + [I- (C^TC)^* (C^TC)]\bm{y}
\end{equation}
其中 $(C^TC)^*$ 是 $C^TC$ 的广义逆,$\bm{y}$ 是任意 $n$ 维向量,故此时 $\bm{x}$ 是不唯一的.

\setcounter{section}{2}
\section*{C}
\begin{equation}
	D = \left[\begin{array}{c|c|c|c}
		I & O & \bm{0} & \bm{0} \\ \hline
		O & I & \bm{0} & \bm{0} \\ \hline
		O & O & \bm{h}_2 & 2\bm{h}_1 \\ \hline
		O & O & \bm{h}_3 & 3\bm{h}_2 \\ \hline
		O & O & \bm{h}_4 & 4\bm{h}_3 \\ \hline
		O & O & \bm{h}_5 & 5\bm{h}_4
	\end{array}\right] \quad B = \left[\begin{array}{c|c|c|c}
		I & \bm{h}_0 & O & \bm{0} \\ \hline
		O & \bm{h}_1 & I & \bm{h}_0 \\ \hline
		\bm{0}^T & 1 & \bm{0}^T & 0 \\ \hline
		\bm{0}^T & 0 & \bm{0}^T & 1
\end{array} \right]
\end{equation}
其中,$O$ 为 $2\times 2$ 的零矩阵,$\det[B]=1$,则有
\begin{equation}
	\begin{split}
			\det[C^T C] =& \det[(DB)^T (DB)]=\det[D^T D] \\
			=&\det \left[\begin{array}{cc}
			\sum_{i=1}^{4} \bm{h}_{i+1}^T \bm{h}_{i+1} & \sum_{i=1}^{4} \bm{h}_{i+1}^T \bm{h}_i \\
			\sum_{i=1}^{4}(i+1)\bm{h}_i^T\bm{h}_{i+1} & \sum_{i=1}^{4}(i+1)^2 \bm{h}_i^T \bm{h}_i
		\end{array}\right]
	\end{split}
\end{equation}
计算得到
\begin{equation}
	\begin{split}
		\det[C^T C] =&\left[\sum_{i=1}^{4}\bm{h}_{i+1}^T\bm{h}_{i+1} \right] \left[\sum_{i=1}^{4}(i+1)^2 \bm{h}_{i}^T\bm{h}_i \right] - \left[ \sum_{i=1}^{4} (i+1) \bm{h}_i^T \bm{h}_{i+1} \right]^2 \\
		=& \sum_{i=1}^{4} \sum_{j=1}^{4} [(i+1)^2 (\bm{h}_{i}^T \bm{h}_i)(\bm{h}_{j+1}^T \bm{h}_{j+1}) - (i+1)(j+1)(\bm{h}_i^T \bm{h}_{i+1}) (\bm{h}_j^T \bm{h}_{j+1})] \\
		=&\frac{1}{2}\sum_{i=1}^{4} \sum_{j=1}^{4}\lbrace [(i+1)f_i f_{j+1} - (j+1)f_j f_{i+1}]^2 + 2[(i+1)f_i g_{j+1} - (j+1) g_j f_{i+1}]^2 \\
		&+ [(i+1)g_i g_{j+1} - (j+1) g_j g_{i+1}]^2 \rbrace
	\end{split}
\end{equation}
根据上式 $\det [C^T C] = 0$当且仅当
\begin{equation}
	\begin{split}
		(i+1)f_i f_{j+1} - (j+1) f_{i+1} f_{j} =& 0 \\
		(i+1) f_i g_{j+1} - (j+1) f_{i+1} g_j =& 0  \\
		(i+1) g_i g_{j+1} - (j+1) g_{i+1} g_j =& 0
	\end{split}
\end{equation}
对所有的 $i,j =1,2,3,4$ 成立.通过利用式(2-9)及求解微分方程得到上式的等价方程
\begin{equation}
	\begin{split}
		2f_1 f_3 - 3f_2 f_2 =& 0 \\
		f_1 g_2 - f_2 g_1 =& 0 \\
		2g_1 g_3 - 3 g_2 g_2 =& 0
 	\end{split}
\end{equation}
观察上式可知 $\bm{h}_0 = [f_0 \quad g_0]^T = c\text{(常数)}$,是该方程的一个平凡解,且得到 $\bm{h}_i = [f_i \quad g_i]^T=\bm{0},i \neq 0$,考虑方程的非平凡解,令 $\bm{h}_1 \not \equiv 0$,由式(4-19)的第二个等式得到
\begin{equation}
	\bm{h}_2 = \lambda \bm{h}_2
\end{equation}  
其中,$\lambda$ 是需要被确定的标量函数,根据式(2-9)
\begin{equation}
	\frac{d\bm{h}_1}{dt} - \lambda \bm{h}_1 =\bm{0}
\end{equation}
上式对 $t$ 在区间 $[t_0,t]$ 上求积分得
\begin{equation}
	\bm{h}_1(t) = \exp \left[ \int_{t_0}^{t_1} \lambda (\mu) d \mu \right] \bm{h}_1 (t_0)
\end{equation}
由式(4-19)第一个和第三个等式得到
\begin{equation}
	2\bm{h}_1^T \bm{h}_3 - 3 \bm{h}_2^T \bm{h}_2 = 0
\end{equation}
对式(4-20)求微分得
\begin{equation}
	\bm{h}_3 = \dot{\lambda} \bm{h}_1 + \lambda \bm{h}_2
\end{equation}
把 $\bm{h}_2,\bm{h}_3$ 代入式(4-23)得
\begin{equation}
	(2\dot{\lambda} - \lambda^2) \bm{h}_1^T \bm{h}_1 = 0
\end{equation}
由于 $\bm{h}_1 \not \equiv 0$,因此有 $\bm{h}_1^T \bm{h}_1 \not \equiv 0$,从而得到
\begin{equation}
	2\dot{\lambda} - \lambda^2 =0
\end{equation}
求解上式
\begin{equation}
	\lambda (t) = \frac{\lambda (t_0)}{1-[\lambda (t_0)/2](t-t_0)}
\end{equation}
再根据式(4-22)解得
\begin{equation}
	\bm{h}_1 = \frac{d \bm{h}_0 (t)}{d t} = \frac{\bm{h}_1 (t_0)}{\left\{1-[\lambda (t_0)/2](t-t_0)\right\} ^2}
\end{equation}
上式对时间在区间 $[t_0,t]$ 求积分得到
\begin{equation}
	\bm{h}_0(t) = \bm{h}_0 (t_0) +\frac{\bm{h}_1(t-t_0)}{1-[\lambda (t_0)/2](t-t_0)}
\end{equation}
上述结果在数学上满足 $\det[C^T C] =0$ 的充要条件,由于 $\bm{h}_0 = [r_x/r_z \quad r_y/r_z]^T$,可得
\begin{equation}
	\left[\begin{array}{c}
		\bm{h}_0 (t) - \bm{h}_0 (t_0) \\ \hline
		0
	\end{array}\right] = \frac{\bm{r}(t)}{r_z (t)} - \frac{\bm{r}(t_0)}{r_z (t_0)}
\end{equation}
由于 $t_0$ 时刻船没有操纵,即 $\bm{a}_0(t_0) = \bm{0}$,对上式求微分
得到
\begin{equation}
	\begin{split}
		\left[\begin{array}{c}
			\bm{h}_1 (t_0) \\ \hline
			0
		\end{array}\right] =& \frac{r_z(t_0)\bm{v}(t_0) - v_z(t_0)\bm{r}(t_0)}{r_z^2 (t_0)} \\
		\left[ \begin{array}{c}
			\bm{h}_2(t_0) \\ \hline
			0
		\end{array}\right] =& -2\frac{v_z(t_0)}{r_z(t_0)}\left( \frac{r_z(t_0) \bm{v}(t_0) - v_z(t_0)\bm{r}(t_0)}{r_z^2(t_0)}\right)
	\end{split} 
\end{equation}
由式(4-20)和上式可得
\begin{equation}
	\lambda (t_0) = -2 \frac{v_z(t_0)}{r_z (t_0)}
\end{equation}
根据式(4-29),式(4-30),式(4-31)和式(4-32)可得到
\begin{equation}
	\frac{\bm{r}(t)}{\bm{r}_z (t)} = \frac{\bm{r}(t_0) +(t-t_0)\bm{v}(t_0)}{r_z(t_0) + (t-t_0) v_z (t_0)}
\end{equation}
最后由式(2-3)可将上式写为
\begin{equation}
	\int_{t_0}^{t} (t-\tau) \bm{a}_0 (\tau)d \tau = \alpha(t) [\bm{r}(t_0) + (t-t_0) \bm{v}(t_0)]
\end{equation}
其中 $\alpha(t)$ 是任意标量函数.由此可知 $\det[C^T C] \not \equiv 0$ 等价于
\begin{equation}
	\int_{t_0}^{t} (t-\tau) \bm{a}_0 (\tau)d \tau \not \equiv \alpha(t) [\bm{r}(t_0) + (t-t_0) \bm{v}(t_0)]
\end{equation}