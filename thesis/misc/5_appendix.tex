%%
% The BIThesis Template for Bachelor Graduation Thesis
%
% 北京理工大学毕业设计(论文)附录 —— 使用 XeLaTeX 编译
%
% Copyright 2020 Spencer Woo
%
% This work may be distributed and/or modified under the
% conditions of the LaTeX Project Public License, either version 1.3
% of this license or (at your option) any later version.
% The latest version of this license is in
%   http://www.latex-project.org/lppl.txt
% and version 1.3 or later is part of all distributions of LaTeX
% version 2005/12/01 or later.
%
% This work has the LPPL maintenance status `maintained'.
%
% The Current Maintainer of this work is Spencer Woo.
%
% Compile with: xelatex -> biber -> xelatex -> xelatex

\addcontentsline{toc}{chapter}{附~~~~录}
\chapter*{\vskip 10bp \textmd{附~~~~录} \vskip -6bp}

\section*{A}
由式(2-4)得到
\begin{subequations}
	\begin{align}
		0=& r_x\sin\beta - r_y\cos\beta \\
		0=& r_x\sin\varepsilon\cos\beta+r_y\sin\varepsilon\sin\beta -r_z \cos\varepsilon
	\end{align}
\end{subequations}
令式(4-1a)等式两边同时乘上 $\sin\beta$,式(4-1b)等式两边同时乘上 $\cos\beta/\sin\varepsilon$,再将结果相加得到
\begin{equation}
	0=r_x - r_z \cos\beta\cot\varepsilon
\end{equation}
再令式(4-1a)等式两边同时乘上 $-\cos\beta$,式(4-1b)等式两边同时乘上 $\sin\beta/\sin\varepsilon$,将结果相加得到
\begin{equation}
	0=r_y - r_z\sin\beta\cot\varepsilon
\end{equation}
从而得到
\begin{equation}
	H \bm{X} = \bm{0}
\end{equation}

\section*{B}
考虑如下线性系统
\begin{equation}
	\begin{split}
		\dot{\bm{x}} =& A\bm{x} - \bm{w} \\
		\bm{z} =& H \bm{x}
	\end{split}	
\end{equation}
其中 $A,H$ 是给出的连续可微矩阵,$\bm{w}$ 是已知向量,$\bm{x}$ 是 $n$ 维向量,$\bm{z}$ 是测量向量,并且该系统中不存在噪声干扰,视为理想情况。
令 $C_0 = H, \bm{z}_0 = \bm{z}$,第一次微分可得
\begin{equation}
	\dot{\bm{z}}_0 = \dot{C}_0\bm{x} + C_0\dot{\bm{x}}
\end{equation}
将式(4-5)中 $\dot{\bm{x}}$ 代入上式,并令 $C_1=\dot{C}_0 + C_0A,\bm{z}_1=\dot{\bm{z}}_0 + C_0\bm{w}$ 可得
\begin{equation}
	\bm{z}_1 = C_1 \bm{x}
\end{equation}
对式(4-7)微分并重复上述过程可得
\begin{equation}
	\bm{z}_2 = C_2 \bm{x}
\end{equation}
其中 $C_2 = \dot{C}_1 + C_1 A,\bm{z}_2 = \dot{\bm{z}}_1 + C_1\bm{w}$,继续并重复上诉过程 $n-2$ 次得到一组独立的关于 $x$ 的线性方程
\begin{equation}
	\bm{Z} = C\bm{x}
\end{equation}
其中
\begin{equation}
	\begin{split}
		\bm{Z} =& [\bm{z}_0,\bm{z}_1,\cdots,\bm{z}_{n-1}]^T \\
		\bm{z}_0 =& \bm{z} \\
		\bm{z}_{i+1} =& \dot{\bm{z}}_{i} + C_i\bm{w} \quad i=0,1,\cdots\\
		C =& [C_0,C_1,\cdots,C_{n-1}]^T \\
		C_0 =& H \\
		C_{i+1} =& \dot{C}_i + C_i A \quad i=0,1,\cdots
	\end{split}
\end{equation}
对式(4-9)左乘 $C^T$ 可得
\begin{equation}
	C^T\bm{Z} = C^TC \bm{x}
\end{equation}
则当且仅当 $C^TC$ 是非奇异的,即 $\det[C^TC] \not \equiv 0$ 时,$\bm{x}$ 可以被唯一确定。若 $C^TC$ 是奇异的,则
\begin{equation}
	\bm{x} = (C^TC)^*C^T Z + [I- (C^TC)^* (C^TC)]\bm{y}
\end{equation}
其中 $(C^TC)^*$ 是 $C^TC$ 的广义逆,$\bm{y}$ 是任意 $n$ 维向量,故此时 $\bm{x}$ 是不唯一的。