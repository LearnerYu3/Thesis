%%
% The BIThesis Template for Bachelor Graduation Thesis
%
% 北京理工大学毕业设计(论文)第一章节 —— 使用 XeLaTeX 编译
%
% Copyright 2020 Spencer Woo
%
% This work may be distributed and/or modified under the
% conditions of the LaTeX Project Public License, either version 1.3
% of this license or (at your option) any later version.
% The latest version of this license is in
%   http://www.latex-project.org/lppl.txt
% and version 1.3 or later is part of all distributions of LaTeX
% version 2005/12/01 or later.
%
% This work has the LPPL maintenance status `maintained'.
%
% The Current Maintainer of this work is Spencer Woo.
%
% 第一章节

\chapter{引言}

\section{跟踪定位背景介绍}
定位技术在现代社会中被广泛应用,对人们的日常生活、勘探作业以及国防事业等都具有重要意义。随着定位技术的发展也产生了不同方式的定位技术,如无线传感网络(WSN)定位技术\cite{wls1}\cite{wls2}\cite{wsl3}\cite{wsl4},通过多个传感器组成的无线通信网络在其监控的区域内实现目标的定位,可以应用于事件检测(火灾、洪水、冰雹),监视(工业,农业)等,定位算法也比较完善。水声定位系统\cite{bls5}\cite{bls6}\cite{bls7}\cite{bls8},利用水声探测装置通过测量声源到基元之间的相关信息量实现对声源的定位,根据定位系统的基阵长度可以分为:长基线、短基线和超短基线,通常用于对水下目标的勘察。

除此之外,无源定位与跟踪定位技术在国防事业上的应用具有更为重要的意义\cite{9},按照不同定位方法进行划分,无源定位可以分为基于测向的测角定位法、基于测信号到达时间差的时差定位法、基于相对运动产生多普勒频率的频率定位方法,以及这些方法的混合方法。按照定位系统的观测站数量又可划分为单站定位和多站定位。一方面,由于多站定位会存在站间通信稳定、站间信号匹配等方面的问题,而单站定位有着较强的机动性和适应性,另一方面,在固定多站或运动多站的设备受到影响时,定位结果也会受到影响,此时单站定位可以作为补充手段。因此,单站无源定位与跟踪技术的发展显得尤为重要
\section{测向定位法(BO)及算法简述}
单站无源定位跟踪技术是利用一个观测站对目标进行无源定位的技术,由于获得的信息量相对较少,所以实现定位的难度相对较大。在定位时,由观测站对目标进行连续的测量,在获得一定的定位信息后,经过适当的数据处理获得目标的定位信息。

单站跟踪定位在很多应用场合目标是运动的,对于固定的目标观测站可通过连续的运动获得目标的角度信息,再通过三角定位法对其定位,该问题相对简单。本文考虑的场景为运动的目标,根据文献\cite{10}的第8章,利用对运动的目标的测量信息有可能获得目标位置和运动状态的过程称为目标运动分析(Target Motion Analysis,TMA)。按照观测其的运动方式,TMA可分为两类,第一类观测站自身是机动的,第二类观测站无机动运动。对两类观测站应用不同的跟踪定位方法可能会得到不同的定位结果。目前,单站无源跟踪定位技术采用的方法主要有\cite{11}:到达时间定位法(TOA)、多普勒频率定位法、测向定位法(BO),以及联合信息定位法。然而对于运动的目标只测量单一的信息量通常并不能得到目标的全部运动状态,通常需要联合两种或多种信息量以实现对目标的测量。不过对观测站的运动状态做相应的要求也可以实现对特殊运动状态的目标进行定位跟踪。其中测向定位法的研究较为完善。

测向定位法是单站无源跟踪定位技术中研究最多最主要的一种方法,测向定位法是指仅利用目标相对于观测站的角度信息对目标进行定位。当观测站固定时,文献\cite{10}中8.4节指出对于二维平面上匀速直线运动的目标,只要其运动方向不是径向的,就可以通过三次以上的测量确定目标的运动方向,但是不能得到目标的距离及速度大小。文献\cite{12}在测角信息的基础上添加径向加速度信息,得出对于非径向运动的匀速运动目标,利用径向加速度信息和角度信息可以实现对目标的定位,并且目标距离越近,相对运动速率越大运动方向越接近切向,其可观测度越高,并且由于观测站与目标轨迹可构成二维平面,故该结论可以推到三维空间。

对于运动的观测站,实质是利用多次观测来拟合目标的运动轨迹,文献\cite{bo1}\cite{bo2}给出了目标可观测性条件,其中文献\cite{bo1}给出了二维平面上匀速直线运动目标可观测的充要条件,文献\cite{bo2}给出了三维空间上匀速直线运动目标可观测的充要条件。因此对于满足条件的机动观测站可实现对匀速目标的定位跟踪。

对于纯角度单站无源跟踪定位的算法主要有最小二乘法(LS)、极大似然估计法(MLE)、、近似无偏估计法(AUE)以及卡尔曼滤波法(KF)。文献\cite{15},\cite{wls2}通过对角度信息量的处理,将非线性的测量方程转化为线性方程求解,由此得到了处理角度信息的最小二乘法。文献\cite{16}推导了二维平面上匀速目标基于奇异值分解(SVD)的总体最小二乘法。文献\cite{15}通过对角度信息的噪声概率密度函数分析得到其极大似然估计函数,并在考虑加权对位置定位的影响时得到了极大似然估计的迭代算法。文献\cite{10}中第9章中推导了二维平面上基于纯角度信息对匀速运动目标进行定位的极大似然估计的迭代算法。由于在考虑噪声误差的情况下,最小二乘法是有偏估计,文献\cite{17}提出了二维平面上基于角度的双站无源定位近似无偏估计,与扩展卡尔曼滤波法相比,该方法的收敛速度和误差精度都相对较好。文献\cite{18}提出了基于角度和时差的双站定位的近似无偏估计算法,结果表明与有偏估计算法相比该算法具有更高的精度。文献\cite{19}给出了卡尔曼滤波和扩展卡尔曼滤波的方法和原理,卡尔曼滤波对模型线性的要求较高,无法直接应用于非线性无源定位系统。通过对卡尔曼滤波加以扩展和改进可以得到适应于非线性模型的无源定位系统,扩展卡尔曼滤波(EKF)即可用于解决非线性的无源定位模型。不过在将非线性模型线性化的过程中通常会带来误差,导致收敛不稳定,甚至发散,使其应用受限。文献\cite{20}基于扩展卡尔曼滤波提出了一种改进算法,该算法可以消除不利误差的影响,并提高定位精度。不过卡尔曼滤波算法计算量相对较大,并且对迭代初值的依赖较大,通常会出现不收敛的情况。

本文研究的问题是基于纯角度的单站跟踪定位算法,并根据上述相关文献推导了三维空间上基于纯角度的匀速运动目标的跟踪定位的相关算法,包括最小二乘法、总体最小二乘法、极大似然估计法和近似无偏估计法。最后通过数值实验比较四种算法的性能。
