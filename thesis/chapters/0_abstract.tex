%%
% The BIThesis Template for Bachelor Graduation Thesis
%
% 北京理工大学毕业设计(论文)中英文摘要 —— 使用 XeLaTeX 编译
%
% Copyright 2020 Spencer Woo
%
% This work may be distributed and/or modified under the
% conditions of the LaTeX Project Public License, either version 1.3
% of this license or (at your option) any later version.
% The latest version of this license is in
%   http://www.latex-project.org/lppl.txt
% and version 1.3 or later is part of all distributions of LaTeX
% version 2005/12/01 or later.
%
% This work has the LPPL maintenance status `maintained'.
%
% The Current Maintainer of this work is Spencer Woo.

% 中英文摘要章节
\topskip=0pt
\zihao{-4}

\vspace*{-7mm}

\begin{center}
  \heiti\zihao{-2}\textbf{\thesisTitle}
\end{center}

\vspace*{2mm}

\addcontentsline{toc}{chapter}{摘~~~~要}
{\let\clearpage\relax \chapter*{\textmd{摘~~~~要}}}
\setcounter{page}{1}

\vspace*{1mm}

\setstretch{1.53}
\setlength{\parskip}{0em}

% 中文摘要正文从这里开始
单站无源跟踪定位是一项重要的定位技术,可以补充多站跟踪定位在灵活性方面的不足,但是单站定位通常得到得观测量数据比多站定位少,因此定位难度较高。本文主要研究基于纯角度的单站无源跟踪定位算法,利用测向定位法的原理实现对目标的定位跟踪。但是在观测站运动的过程中方位角和仰角的变化幅度很小,容易受到噪声的影响,导致定位误差较大。本文考虑普通最小二乘法(OLS)、总体最小二乘法(TLS)、极大似然估计法(MLE)、渐进无偏估计(AUE)四种算法对模型进行求解。比较算法的性能,并探究影响定位结果的因素。最终得出AUE算法是精确度最好的一种算法,并且得到当目标与观测站的距离越近,定位的精确度越高。方位角和仰角做为影响定位结果的两种重要因素,变化幅度越大,定位的精确度越高。

\vspace{4ex}\noindent\textbf{\heiti 关键词:单站无源跟踪定位,测向定位,OLS,TLS,MLE,AUE}
\newpage

% 英文摘要章节
\topskip=0pt

\vspace*{2mm}

\begin{spacing}{0.95}
  \centering
  \heiti\zihao{3}\textbf{\thesisTitleEN}
\end{spacing}

\vspace*{17mm}

\addcontentsline{toc}{chapter}{Abstract}
{\let\clearpage\relax \chapter*{
  \zihao{-3}\textmd{Abstract}\vskip -3bp}}
\setcounter{page}{2}

\setstretch{1.53}
\setlength{\parskip}{0em}

% 英文摘要正文从这里开始
Single-station passive tracking and positioning is an important positioning technology that can supply the flexibility of multi-station tracking and positioning, but single-station positioning usually obtains less observational data than multi-station positioning, so single-station positioning is more difficult. This paper mainly studies the single-station passive tracking and positioning algorithms based on bearing-only, using the principle of directional cosine intersection positioning method to achieve the tracking and positioning of target. however, the azimuth and elevation angles change very little during the movement of observation station, which is susceptible to the influence of noise, leading to large positioning errors. This paper considers four algorithms, Ordinary Least Squares(OLS), Total Least Squares(TLS), Maximum Likelihood Estimation(MLE), and Approximate Unbiased Estimation(AUE), to solve the model. Comparing the performance of the algorithms and exploring the factors that affect the positioning results. It is concluded that the AUE algorithm is the one with the best accuracy, and it is obtained that the closer the distance between the target and the observing station, the higher the positioning accuracy.  Azimuth angle and elevation angle are two important factors that affect the positioning result. The bigger the range of change, the higher the accuracy of positioning results.

\vspace{3ex}\noindent\textbf{Key Words: Single-station Passive Tracking Positioning,Directional Cosine Intersection Positioning,OLS,TLS,MLE,AUE}
\newpage
